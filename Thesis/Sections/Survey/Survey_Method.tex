\section{Method}
\label{section:survey_method}

We tested the validity of the prototype using a survey.
If a survey is not well designed, then it could lead to invalid or irrelevant outcomes.
This chapter describes the design and procedure of the survey.
Additionally, it outlines any threats to its validity.
Our choice of survey technique is a questionnaire.

\subsection{Questionnaire Design}
To design the survey of our prototype, we followed the following rules derived from the works of Bryman\cite{bryman2016social} and de Vaus\cite{de2013surveys}.

\begin{itemize}    
    \setlength\itemsep{0em}
    \item \emph{Introduction:} 
        We devised a clear introduction to describe the research.
    \item \emph{Existing work:} 
        We considered existing questions.
        Regarding projectional editing, we requested the original questionnaires from three papers\cite{meacham2020adaptivevle_SLR,berger2016efficiency, voelter2014towards} about tools developed using projectional editing.
        Unfortunately, none of the original questions were available for assessment.
    \item \emph{Question in mind:} 
        We had the specific research question, ``Which projections can help developers get appropriate feedback about rules?'' in mind when formulating the questions.
    \item \emph{Succinct:} 
        The pool of Drools users that we were personally in contact with was tiny.
        Thus, we had to rely on responses from strangers.
        For this reason, we tried to make the questionnaire as quick to finish as possible.
        This constraint meant we looked particularly hard at removing questions that did not help us to our research goal.
    \item \emph{Pilot:} 
        We piloted the questionnaire with both ourselves and our industrial supervisor. 
        The result of this pilot led to more explanative text before the questions.
    \item \emph{Clarity:} 
        The instructions to each of the questions were tested for clarity by a non-technical third party.
        We took care to rework questions that were long, ambiguous, general, or leading not to be so.
        We also took care to remove jargon, negative wording, and questions that asked about more than one thing.
    \item \emph{Closed questions:} 
        The only open questions were ones from which we wished to extract sentiment.
        To avoid binary questions, where appropriate, we applied a Likert scale\cite{likert1932technique}.
    \item \emph{Single page questions:} 
        Thanks to the UI of Survey Monkey, no questions spanned multiple pages.
    \item \emph{Important questions first:} 
        We started with the research-based questions, leaving the socio-demographic questions, such as skill level, to the end.
\end{itemize}

Appendix \ref{Appendix:Questionaire_text} shows the questionnaire we designed following these principles.


\subsection{Participants}

The requirement for participants is that they have at least a little experience with using Drools.
We hoped to get a statistically significant number of participants.

\subsection{Validity}
We addressed the non-response bias\cite{armstrong1977estimating} by making the questionnaire short and easy to answer.
Because of the nature of the participant selection for this survey, it will be challenging to address the self-selection bias caused by the voluntary nature of the response.

Common method bias, i.e., ``variance that is attributable to the measurement method rather than to the construct the measures represent''\cite{podsakoff2003common} can be responsible for 25\% or more of variable relational influence.
As we are only conducting a single survey, we will not be able to prevent this. 
However, we took the following small precautions.
We tested the survey to remove question ambiguity, mood influences, and length issues.
We mixed the order of questions in the survey to mitigate the issues caused by the similarity, proximity, and location of items.
We varied the scales and order of our Likert scales.

The main statistical methods to address common method bias, i.e., ``Harman's single factor test''\cite{podsakoff2003common} and the ``marker variable''\cite{lindell2001accounting} have been found to be lacking in grounding\cite{gorrell2011countering}.
The marker variable approach is appropriate if used with caution.
However, it may not be possible to gain a statistically significant outcome with our expected response size.

\subsection{Pre-test}
We sent our first pass of the survey to our industrial supervisor, who has experience with Drools.
With this pre-test, we hoped to remove ambiguously worded or leading questions.
Additionally, we wanted to confirm that the questionnaire took around 10 minutes to complete.

As a result of this pre-test, we updated much of the explanatory text.

\subsection{Sampling}
Within our professional network, we only had a connection with very few Drools developers.
We also considered that having acquaintances answer the questionnaire could introduce some biases we could not account for.
So, we had to expand our sampling reach.

Our first approach was to search StackOverflow for question askers and answerers about Drools.
Our preference was to find email addresses, failing that Twitter contacts.
Unfortunately, this proved quite limited.
We only managed to harvest 13 email addresses, and six Twitter handles.

Our following approach was to interrogate our LinkedIn connections for anyone who claimed Drools as one of their marketable skills.
At two degrees of separation in our LinkedIn network, we found 204 candidates with Drools skills listed.
From these, we harvested 54 email addresses, and 40 Twitter handles.

We chose not to expand our search to three degrees of separation.
At two degrees of separation, we share a relationship with the same person.
We thought it would be harder to take advantage of the social pressure to answer when we do not have that shared relationship.

\subsection{Procedure}

The questions, as described in appendix \ref{Appendix:Questionaire_text}, were turned into a survey using the Survey Monkey service.
We also show screenshots of one version of the questionnaire in appendix \ref{Appendix:Questionaire_text}.

To encourage response, we crafted a short introduction email, heavily relying on techniques designed to enhance response as discussed by  Cialdini\cite{goldstein2008yes}.
