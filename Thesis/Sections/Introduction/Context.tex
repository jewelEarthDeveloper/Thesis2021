\section{Project Context}

Khonraad Software Engineering, a subsidiary of Visma, hosted this investigation.
Khonraad provides mission-critical services focussed on the automation of workflows at the cross-section of local government and healthcare.
Specifically, Khonraad facilitates the mental health care and coercion laws in the Netherlands - WVGGZ, WZD, and WTH - which allow agencies to intervene in domestic violence, psychiatric disorders, and illnesses.

Khonraad's system facilitates reporting and communication between municipalities, police, judiciary, lawyers, mental health care, and many social care institutions.
The system has 15,000 users and is available 24/7.

Configuration and administration require complex matrices of compliance mechanisms, access user rights and communication settings.
The sensitivity of the personal data, being both medical and criminal, means security is of utmost importance.
The security against data loss, preventing unlawful disclosure and guaranteeing availability, especially during crises, is crucial.
Demonstration of the correctness of the configuration is a significant concern in the company.

The work environment at Khonraad allows us to work on an existing project, where success will impact the lives of those in critical need. 
Khonraad software contains Drools code that has evolved and grown together with the changes in the laws. 
Working in this environment means that we are also dealing with real-life issues and not just thought experiments.
