\section{Project context}

This investigation was hosted by Khonraad Software Engineering, a subsidiary of Visma.
Khonraad provides mission-critical services focussed on the automation of workflows at the cross-section of local government and healthcare.
Specifically, Khonraad facilitates the mental health care and coercion laws in the Netherlands - WVGGZ, WZD, and WTH - which provide agencies the ability to intervene in domestic violence, psychiatric disorders, and illnesses.

Khonraad's system facilitates reporting and communication between municipalities, police, judiciary, lawyers, mental health care, and many social care institutions.
The system has 15,000 users and is available 24/7. 

Configuration and administration use complex matrices of compliance mechanisms, access user rights and communication settings.
The sensitivity of the personal data, being both medical and criminal, means security is of utmost importance.
The security against data loss, preventing unlawful disclosure and guaranteeing availability, especially during crisis situations, is crucial.
Demonstration of the correctness of the, often changing, configuration is a major concern in the company. 

This work environment allows us to work on an existing project, where the tangible success will have an impact on the lives of those in critical need.
Khonraad has it's own implementations in the Drools language, that have evolved over the iterations of the laws.
The evolution of the code base over the years means that the real-life issues we came across are not just thought experiments.