\section{Research Method}
\label{section:Research_Method}

To answer the research questions from section \ref{section:Research_Questions}, we pursued three approaches.

Our chosen path to answering RQ 1, ``What is the current state of language workbenches supporting projectional editing?'' was to conduct a systematic literature review (SLR).
The SLR aims to answer this question by interrogating current research with the sub-questions, ``Is there any current research in the area of projectional editing?'',  ``Which tools are currently being used for research?'' and ``What is the sentiment in papers currently discussing projectional editing?''.
We describe how we went about this in section \ref{section:slr_method}. 

This SLR showed that whilst still niche, projectional editing is maturing, with one successful language workbench dominating the commercialisation of this field.

We attacked RQ 2, ``Which projections can we create to help developers get appropriate feedback about rules?'' by implementing limited versions of the Drools language using the MPS language workbench and creating projections on top of this.
We first created a minimal pilot version, called Really Simple Drools, which validated that this approach would work.
Next, we created Drools-Lite, which had the look and feel of traditional Drools and enough functionality that we could present it to experienced Drools users.
We describe this design science research (DSR) approach in detail in section \ref{section:dsr_method}.

The outcome of this research was several projections that we felt would improve the understanding of larger collections of Drools rules.

Finally, with a survey, we tackled RQ 3, "Do projections of Drools business rules lead to greater understanding of those rules?".
We presented the results of RQ2 to experienced industrial and academic Drools users, as laid out in section \ref{section:survey_method}.
We compared some of our projections with textual projections similar to the code they were used to.

The survey showed that whilst there was interest in the approach and that some projections were more understandable than others, textual presentations, at least for this population, were still considered easier to understand.