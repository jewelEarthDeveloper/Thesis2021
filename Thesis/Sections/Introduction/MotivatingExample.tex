\section{Problem statement}

Miller's Law\cite{miller1956magical} states that an average human can hold in his short-term memory 5-9 objects.
This is often an argument for more succinct code.
The argument being anything that is not immediately in the developers vision has to be stored in her memory.
With it being impractical to reason about code that she cannot recall, then the fewer relevant items to her reasoning that are out of view the easier it is to reason about the code.

Our host organization, Khonraad Software Engineering, a subsidiary of Visma, provides mission-critical services focussed on the automation of workflows at the cross-section of local government and healthcare.
Specifically, Khonraad facilitates the mental health care and coercion laws in the Netherlands - WVGGZ, WZD, and WTH - which provide agencies the ability to intervene in domestic violence, psychiatric disorders, and illnesses.

Khonraad's system facilitates reporting and communication between municipalities, police, judiciary, lawyers, mental health care, and many social care institutions.
The system has 15,000 users and is available 24/7.

Configuration and administration use complex matrices of compliance mechanisms, access user rights and communication settings.
The sensitivity of the personal data, being both medical and criminal, means security is of utmost importance.
The security against data loss, preventing unlawful disclosure and guaranteeing availability, especially during crisis situations, is crucial.
Demonstration of the correctness of the, often changing, configuration is a major concern in the company.

This configuration is done in a business rule system, specifically JBoss Drools.

Drools is a language that shares an unfortunate characteristic with many other rules languages.
It is verbose and can contain many rules that can interact with each other without obvious visual connection.
As Forgy\cite{forgy1989rete} points out, for production systems in general, ``production systems have another property that makes them particularly attractive for constructing large programs: they do not require the programmer to specify in minute detail exactly how the various parts of the program will interact''.
This property leads to very large and hard to reason about collections of implicitly connected rules.

Reasoning over a small number of rules is already surprisingly hard.
Our host organization has many rules and, thus, reasoning about them is particularly challenging.

We have observed the difficulty that developers have trying to reason about and edit collections of Drools files.
We hypothesize that developers can be presented with different views on their code that will allow them to better understand the code.
The problem we wish to solve - how to improve the ability to reason about large collections of Drools rules - we believe, lends itself to the technique of projectional editing.
By using projections to improve feedback whilst coding, we believe that this can reduce the representation impedance mismatch that hampers developer's reasoning.

The problem considered in this thesis is how to present rules to a developer in a way in which she can interact with [TODO: finish this thought].
As is perhaps already obvious it is not our intention to override the will of the language engineers who have spent many years developing this language and it's ecosystem.
The goal of this thesis is to augment the current developer experience.