\section{Method: Systematic Review}\label{section:Method_systematic_review}

To answer the first Research Question, ``what is the current state of Projectional Editing?'', we conducted a systematic literature review.
Hereafter, we describe the method we undertook.

To carry out this review we followed Kitchenham's\cite{Kitchenham_2015} advice on systematic review protocol validation, (see appendix \ref{appendix:protocolValidation} for the exact checklist we used).

\subsection{Motivation}
The motivation that preceded this research was a requirement to understand if projectional editing was an idea that was worth investigating.
Through our background research we saw an interest in the precursors to projectional editing in the late 70's through to the mid 80's.
This seemed to be abandoned until the mid 90's following Charles Simonyi's treaties on Intentional programming.
This did not lead to a swell in academic research as his companies product, Intentional Domain Workbench was a closed commercial product.
There seemed to be a burst of academic interest after the release of JetBrain's OpenSource Meta Programming System (MPS) in the late 2000's.

Is there a need for a study of this topic? 
We believe, at least in the microcosm of this master's project it is helpful to know whether we are researching in an area that is dying of vibrant.

For the wider community, there does not seem to be any systematic reviews specifically about projectional editing, let alone recently.
This study is not extending any previous Systematic Review as, although there were literature surveys and mapping studies covering some adjacent fields, no SLRs were found.
Thus we believe it may be useful for those in the language engineering research community to bring together all current research in the area of projectional editing in one place.

\subsection{Research Question}
This paper we only have one research question to synthesize the findings of scientific papers towards.
This is ``What is the current state of Projectional Editing?''

This question for us can be broke down into:
\begin{itemize}
    \item \textbf{Sub Question 1} ``Is there current research in the area of projectional editing?''
    \item \textbf{Sub Question 2} ``What tools are currently being used for research?''
    \item \textbf{Sub Question 3} ``What is the sentiment in papers currently discussing projectional editing?''
\end{itemize}


\subsection{Search Strategy}

The search process is automated as SLRs require a high level of completeness, which cannot be effectively achieved manually.
Our first major decision was whether to engage in creating a quasi-gold standard as advised by Zhang\cite{Zhang_2011}.
Zhang noted that the ad-hoc nature of search strategies in SLRs has limitations.
We executed a preliminary ad-hoc search to try and ascertain the extent of the research space.
Upon satisfying ourselves that it was small enough, we rejected the Quasi-Gold standard as overkill for our requirements.

The search terms we landed on were as follows:
{\obeylines\obeyspaces
\texttt{
    ``PROJECTIONAL EDITING'' 
       OR 
    ``PROJECTIONAL EDITOR'' 
}}

This is to be adjusted to fit the query syntax of the various search engines.

As most Research Search engines offer the option of date ranges, and to save the effort of excluding later we also used the date range to eliminate unnecessary paper at the automated search stage.
Our research question we are specifically looking at the current state of projectional editing.
A design decision of many research search engines is that date ranges can often only be defined in whole years.
When designing our search strategy, it was near the beginning of 2021, and thus we feared that this would be too small a search space, thus we set our date range to be from the beginning of 2020 to present.
For the sake of reproducibility, it is advised to remove any papers after 31st July 2021.

The Search Engines used are shown in table \ref{table:searchEngines}.

\begin{table}[H]
    \begin{center}         
        \begin{tabular}{|l||l|}
            \hline
            ACM digital library       & Google Scholar       \\
            \hline
            BASE                      & CORE                 \\
            \hline  
            IEEE Xplore               & ISI Web of Science   \\
            \hline  
            Microsoft Academic        & Science.gov          \\
            \hline  
            Wiley InterScience        & SCOPUS               \\
            \hline  
            Semantic Scholar          & SpringerLink         \\
            \hline  
        \end{tabular}
    \end{center}
    \caption{Search engines used}
    \label{table:searchEngines}
\end{table}

Once we have filtered the automated search through the criteria of the selection stage, we will use that as our starting set for snowballing.
Our filtering will be done before the quality of the papers has been assessed, as we feel that excluding papers on quality of primary study issues may artificially limit the network of potential papers.
Our snowballing procedure shall follow the advice of Wohin\cite{Wohlin_2014}.
This is the idea of using the reference lists from our start set and applying the same selection criteria to these.

Where possible we will get the forward snowballing papers from the ``cited by'' functionality of Google Scholar.
Because of the range of the search being to present, all papers that cite the target paper will fall within our criteria.
For backward snowballing we will manually filter the bibliography section of the selected papers, selecting any paper published in 2020 or 2021

After gathering all the papers from the forward and backward snowballing we will again apply the selection criteria.
This process will iterate recursively until no new papers are found.
All of the papers not excluded in each iteration will be the basis for the quality of primary studies filtering stage.

After the final iteration, as a final step the selected papers will have a deeper scan.
This is to verify our initial scan that the papers met our inclusion criteria, before moving on to the quality assessment.

\subsection{Study Selection}

The inclusion criteria are:
\begin{itemize}
       \item Studies are about or mention projectional editing or one of it's synonyms
       \item It is published in during the period 2020\-2021
\end{itemize}

The exclusion criteria are:
\begin{itemize}
       \item Books and grey literature
       \item not English
       \item no full text available
       \item papers with serious issues with grammar or vocabulary
       \item not a previously selected paper
       \item not a paper about a previously reported on study
\end{itemize}

If multiple papers look at the same study with different approaches, then the data will be aggregated in the synthesis stage.

As a lone researcher, we must be aware of bias in positively including relevant papers and excluding irrelevant papers.
We will follow Kitchenham's suggestions to overcome such bias:
\begin{itemize}
       \item Test-retest 
       \begin{itemize}
              \item We will assess the papers once (on title abstract and keywords) against the inclusion and exclusion criteria.
              \item Save all the suggested results
              \item Assess the papers again three days later in a different order to the first  
       \end{itemize}
       \item If there are disagreements, we will use Cohen's Kappa agreement statistic\cite{Cohen_1960} to see if the process needs to be refined.
\end{itemize} 

If our searches appear to be too large for a lone researcher, we will turn to text mining.
We will be cautious to use this.
O'Mara-Eves et al.'s systematic review of text mining in systematic reviews\cite{OMara-Eves_2015}, recommends that this can be used for prioritization, but finds that for exclusion screening, although promising, it is not yet proven.

An SLR is interested in studies rather than papers.
There is a many-to-many relation between papers and studies.
We will review the selected papers to note when this has happened in our results to make sure studies do not get over or undercounted.

\subsection{Quality of Primary Studies}
To discover explanatory reasons for why there may be differences in study results, and to weigh how valuable specific studies are, we will assess the quality of the selected studies.

To try and avoid a Results Section bias we will be operating a results-blind quality assessment.
Study quality will be based on the methods section of the papers only.
This bias is threatened because results are summarized in the abstract.
The study quality will not be measured until after the selection process is complete, though it will, in part be carried out before the selection re-test.
list of papers will be randomly sorted before assessing for quality.

For EBM studies there are some well-known hierarchies of evidence for study quality thresholds such as the CRD Hierarchy of Evidence\cite{Cochrane_2019}.
Kitchenham in, Procedures for Performing Systematic Reviews\cite{Kitchenham_2004_2}, suggests the hierarchy shown in table \ref{table:hierarchy}

\begin{table}[H]
    \begin{center}   
    \resizebox{\textwidth}{!}{%
      
        \begin{tabular}{|l||l|}
            \hline
            Rank     & Description                                                                                         \\
            \hline
            \hline
            1        & Evidence obtained from at least one properly designed randomized controlled trial                   \\ 
            \hline  
            2        & Evidence obtained from well-designed pseudo-randomized controlled trials                            \\
                     & (i.e. non-random allocation to treatment)                                                           \\
            \hline  
            3-1      & Evidence obtained from comparative studies with concurrent controls and allocation not randomized,  \\
                     & cohort studies, case-control studies or interrupted time series with a control group                \\
            \hline  
            3-2      & Evidence obtained from comparative studies with historical control, two or more single-arm studies, \\
                     & or interrupted time series without a parallel control group                                         \\
            \hline  
            4-1      & Evidence obtained from a randomized experiment performed in an artificial setting                   \\
            \hline  
            4-2      & Evidence obtained from case series, either post-test or pre-test/post-test                          \\
            \hline  
            4-3      & Evidence obtained from a quasi-random experiment performed in an artificial setting                 \\
            \hline  
            5        & Evidence obtained from expert opinion based on theory or consensus                                  \\
            \hline  
        \end{tabular}}
    \end{center}
    \caption{Study design hierarchy for software engineering}
    \label{table:hierarchy}
\end{table}

These different types of study have different quality assessment criteria.
As the nature of our research questions is not likely to attract randomized or pseudo-randomized controlled trials or experiments, our quality assessment checklists are created with comparative studies and case series in mind.
To assess the strength of each primary study we used the checklists shown in appendix \ref{appendix:QualityAssesmentChecklist}.
The checklists were based on a subset of the questions suggested in Keele's guidelines on SLRs\cite{keele2007guidelines}, which in turn extracted questions from previous mostly medical systematic reviews.
Where necessary the selected questions were modified for software engineering.

These checklists are addressed toward general research.
In Software Engineering many studies that fall under what Gregor\cite{gregor2006nature}, in ``A Taxonomy of Theory Types in Information Systems Research'' calls ``Type V: Theory for Design and Action''.
The checklists do not address this type of research well.
On investigation into how others SLRs conduct quality assessment we did not find a solution to this issue.
Therefore we will continue with the checklists as in the appendix, using the checklists for Case Study for Type V research papers.
We will take this into account before dismissing results of this type on basis of their quality score.

As this study will be carried out by a lone researcher, there is no need to have a process for disagreements.
To check on the bias, several papers will be randomly selected and assessed using the checklist by the academic and the daily supervisors.
Should there be a high disagreement the design of the checklist will be revisited.

The quality checklist is trying to weed out the biases of selection, performance, detection, and exclusion, as well as other threats to the validity of the studies under test.
Validity issues can occur during the design, operation, analysis, or conclusion of an empirical study.

\subsection{Data Extraction}
No data extraction will be necessary for the first sub-question,  ``Is there current research in the area of projectional editing?''.
The fact of the existence of papers that have been verified to be primary studies either into projectional editing theory or practical use of it will be enough to answer the question.

For the question of ``What tools are currently being used for research?'', we shall note each tool discussed specifically with regards to the study being carried out.

Finally, for the sentiment we shall pass each paragraph of the introduction, the discussion and the conclusion through a sentiment analyser and, if the paragraph is pertinent to projectional editing, we will not it's sentiment score.
The sentiment analysis tool we shall use is Microsoft Azure Cognitive Services Text Analytics.
The code to carry out this task is shown in listing \ref{listing:text_analytics}

\begin{lstlisting}[language=Python, caption=Text Analytics code., captionpos=b, label=listing:text_analytics, breaklines=true]

    from azure.core.credentials import AzureKeyCredential
    from azure.ai.textanalytics import TextAnalyticsClient
    
    endpoint = "REPLACE_WITH_CORRECT_ENDPOINT"
    key = "REPLACE_WITH_CORRECT_KEY"
    
    text_analytics_client = TextAnalyticsClient(endpoint=endpoint, credential=AzureKeyCredential(key))
    
    inputfiles = [[ARRAY_OF_FILES_TO_BE_ANALYSED]]
    
    with open('/content/sample_data/sentiment/output_all.txt','a') as outf:
      for sections in inputfiles:
        for section in sections:
          print("Section: {}".format(section),file=outf)
          f = open('/content/sample_data/sentiment/'+section)
          content = f.readlines()
          # for brevity an optimization to deal with 10 document limit is removed
          if len(content) != 0:
            result = text_analytics_client.analyze_sentiment(content, show_opinion_mining=True)
            docs = [doc for doc in result if not doc.is_error]
            for idx, doc in enumerate(docs):
              print("sentiment: {}".format(doc.sentiment),file=outf)
              print("Document text: {}".format(content[idx]),file=outf)
\end{lstlisting}


\begin{table}[H]
	\centering
	\begin{tabular}{|c | l | l | c |} 
		\hline
		\#& Data Type           & Description                                          & RQ     \\ \hline
		\hline
        1 & Study ID            & Unique identifier for the study                      &        \\ \hline
        2 & Title of Study      & The paper name                                       &        \\ \hline
        3 & Year of Publication & Will be either 2020 or 2021                          &        \\ \hline
        4 & Author(s) Names     & Including affiliation                                &        \\ \hline
        5 & Source of Study     & Name Of Online Database/ Digital Library             &        \\ \hline
        6 & Type of Study       & Publication/Conference/Workshop/Symposium            &        \\ \hline
        7 & Name of Venue       & Journal/Conference in which study has been published &        \\ \hline
        8 & Tools in Study      & A list of the tools used                             & RQ 1.2 \\ \hline
        9 & Sentiment           & The sentiment scores from appropriate paragraphs     & RQ 1.3 \\ \hline		
	\end{tabular}	
	\caption{Data extraction form}
    \label{table:Data_Extraction_Form}
\end{table}

\subsection{Data Aggregation and Synthesis}
As explained by Kitchenham\cite{kitchenham2015evidence}, in software engineering, primary studies will tend to be too heterogeneous for any statistical analysis.
Synthesizing outcomes from multiple methods will be difficult.
Thus our synthesis will take a narrative approach.

Narrative synthesis is telling a story of the who, how, and why of the success or otherwise of the research.
For ADR research, focus will be on what will help or hinder the adoption of the implementations.
It will also examine how reliable the results are and relationships between the studies.

