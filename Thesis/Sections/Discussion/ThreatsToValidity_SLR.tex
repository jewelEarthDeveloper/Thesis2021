\section{Discussion - Systematic Literature Review}
We now examine threats to the construct, the internal and external validity of our systematic literature review, its reliability, and areas of improvement.

\subsection{Threats to Validity}  
As discussed in their tertiary study of SLRs da Silva et al.\cite{DaSilvaFabioQ.B2011Syos}, one of the main problems of SLRs in Software engineering is a focus on practice and not experimentation.
Because of the nature of the subject area we will be making this same shortfall. 
We feel that we have fully addressed their other concerns of SLRs not assessing the quality of our primary studies, bad integration and lack of guidelines.

As with all SLRs the main threats to validity are incomplete set of studies, due to an insufficient search strategy, researcher bias in paper selection and inaccuracy in data extraction.
In our study quality assessment we made use of Runeson et al.'s\cite{runeson2009guidelines} four suggested limitations of studies, namely construct validity, internal validity, external validity, and reliability.
It is only fair that we point this towards our own study.

\subsubsection{Construct Validity}
Regarding construct validity, i.e. whether our research questions match the research subjects methods and measures.
Whilst no measurement system is perfect, some are much further from perfect than others.

For the construct to be valid we need to present the best available evidence.
The nature and modernity of the Projectional Editing might mean that there is plenty of good evidence available in Grey literature and Industrial Articles. 
This under representation of actual, but non-academic studies could lead to a false positive or negative for some of the questions, leading to errors in recommendations.

There maybe things that influence the best evidence such as who is funding the study. 
Is a researcher working or consulting at a Projectional Editing product supplier, and will this skew results.

Are some projectional editors being ignored because of the preference for English papers only?
The focus on English language papers might be biased against projectional editors aimed at non-English speaking markets.


\subsubsection{Internal Validity}
Internal validity, or the causal relationships, does one factor cause an effect or are both factors influenced by something unseen.
On the surface this seems like it would not affect and SLR.


Selection bias - ``projectional editing''

\subsubsection{External Validity}
External validity is the ability to generalize the findings.

\subsubsection{Reliability}
Reliability is how the data and the analysis is dependant on specific researchers.
Here we are presented with a very credible threat in that this research was carried out by a single researcher.

Whilst measures were put in place to try and mitigate this, the reliance on a single person's judgement of the underlying studies, leaves the door to bias wide open.
Another threat is the use of thematic review and narrative review. 
Both of these can be subjective and therefore difficult to reproduce.

In a replication the papers may be met in a different order leading to different predominant themes. 
A big thread to reliability was the [TO DO - ].  
This meant that the selection by title and subsequent selection by abstract were so long winded that these took place over multiple days. 
Thus there is an impact of bias in time of day and time in the session that might effect whether a title that could be considered marginal would be in or out.


\subsubsection{Repeatability vs Reproducibility}
Greenhalgh et al.\cite{GreenhalghTrisha2005Eaeo} in their review of where papers come from in systematic reviews found that 24\% of papers come from "personal knowledge or personal contact".
for the sake of reproducibility we decided not to hunt down relevant papers from knowledgeable people in the field, as this would require anyone trying to reproduce this study to ask the same people at the same knowledge level as us.
This would be impossible as the only person we know who could help us with this is our academic advisor, and he (in our future and your present) will have read this paper, changing his knowledge base forever.
by making this chase we risk missing out a quarter of our completeness.

\subsubsection{Method improvement}


