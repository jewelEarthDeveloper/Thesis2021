\chapter{Related Work}
\label{chapter:RelatedWork}

We split related work into two sections. 
Firstly, those relating to the state of projectional editing.
Finally, those that address the understandability of business rules.

\section{The State of Projectional Editing Workbenches}

Whilst we were particularly looking into the current state of projectional editing, there have been previous studies.
The language workbench challenges, which ran from 2011 to 2016, inspired many papers, most notably Erdweg et al.'s summaries of the 2013\cite{erdweg2013state} and the 2015\cite{erdweg2015evaluating} challenges.
These papers investigated the capabilities of LWBs, including projectional LWBs.
Whilst these were interesting from the point of view of capabilities of LWBs. 
They did not touch on market penetration or other usage indicators, and thus the experimental workbenches were lined up equally against the well-used.
Schindler et al.\cite{schindler2016language} looked at the experience of the language workbench challenge from the point of view of a projectional editing LWB, namely MPS.
We addressed the areas where projectional editing LWBs had a significant advantage over the other LWBs. 
We feel it is a shame that the LWB challenge is no longer occurring.

There does not seem to be much analysis of what is going on in projectional editing currently, except when there is an intersection with adjacent fields such as Model-Driven Approaches, Low-code, Language Orientated Programming, or language workbenches.
We found one systematic mapping study of LWBs\cite{iung2020systematic} from 2020 that again looked at the entire field of LWBs, including the projectional ones.
Whilst looking at LWB features, it also extended into areas such as domains of use.

Regarding MPS, papers from earlier in the last decade tended to concentrate on improving the experience of using MPS, prototypes or new products.
Many of these are referenced in sections \ref{section:projectional_advantages} and \ref{section:projectional_disadvantages}.
Today, as mentioned in papers discussed in section \ref{section:slr_analysis}, research involving MPS tends to be maturing into industrial use of the product and using it in teaching.

\section{Understandability of Business Rules}  

A different approach to ours was taken in the VODRE project\cite{lapaev2014vodre}.
In place of visualising how the rules belonged together, it visualised how they execute.
In optical design, they showed the execution paths that the expert systems used to draw their conclusions.

In a similar fashion to the Drools DMN, Ostermayer et al.\cite{ostermayer2013simplifying} attempt the generation of rules using templating and an external editor.
Along the same lines, the G-AMC tool\cite{sa2016g} created a front end for developing rules.
However, in this case, in the domain of access control management.
These both face the issues we were trying to overcome by separating the tool and the language. 
They also focus on the generation of rules rather than the understanding of them.
