\chapter{Related Work}\label{chapter:RelatedWork}
We split related work into three sections. 
Firstly, those relating to the state of projectional editing.
Next, those that address understandability of business rules.
Finally, work on interesting projections.

\section{The state of projectional editing workbenches}

Whilst we were particularly looking into the current state of projectional editing, there have been previous studies.
The language workbench challenges, which ran from 2011 to 2016, inspired many papers, most notably Erdweg et al.'s summaries of the 2013\cite{erdweg2013state} and 2015\cite{erdweg2015evaluating} challenges.
These papers looked into the capabilities of LWBs including projectional LWBs.
Whilst these were interesting from the point of view of capabilities of LWBs, they did not touch on areas such as market penetration or other indicator of usage, and thus the experimental were lined up against the well used.
Schindler et al.\cite{schindler2016language} looked at the experience of the language workbench challenge from the point of view of a projectional editing LWB, namely MPS.
This paper addressed the areas where projectional editing LWBs had a significant advantage over the other LWBs. 
We feel it is a shame that the LWB challenge is no longer occurring.

There does not seem to be much analysis of what is going on in projectional editing at the moment, except when there is an intersection with adjacent fields such as Model Driven Approaches, Low-code, Language Orientated Programming, or language workbenches.
We found one systematic mapping study of LWBs\cite{iung2020systematic}, from 2020, that again looked at the entire field of LWBs including the projectional ones.
Whilst looking at LWB features it also extended into areas such as domains of use.

With regards to MPS in particular, while there is no mapping of the state of the usage, we saw that papers earlier in the last decade tended to concentrate on how to improve the experience of using MPS, prototypes or new products, for example \cite{pavletic2013extensible,voelter2014generic,voelter2015using,voelter2010language2,voelter2013mbeddr,voelter2013requirements,voelter2014projecting,voelter2015towards,voelter2010embedded,voelter2011product,ratiu2012implementing}.
Today, research involving MPS tends to be maturing into industrial use of the product and using it in teaching, e.g. \cite{prinz2021teaching,voelterdomain_SLR, schindler2021jetbrains_SLR,simi2021learning_SLR,barash2021teaching_SLR,stotz2021migrating_SLR,ratiu2021use}.

\section{Understandability of Business Rules}

A different approach to ours was taken in the VODRE project\cite{lapaev2014vodre}.
In place of visualising how the rules belonged together, it visualized how they executed.
In the domain of optical design, they showed the execution paths that the expert systems used in coming to their conclusions.

In a similar fashion to the drools DMN, Ostermayer et al.\cite{ostermayer2013simplifying}, attempt the generation of rules using templating and an external editor.
Along the same lines, the G-AMC tool\cite{sa2016g} created a front end for developing rules, however in this case in the domain of access control management.
These both face the issues we were trying to overcome with a separation between the tool and the language. 
They are also focused on the generation of rules rather than the understanding of them.

\section{Interesting projections}
Some of the most interesting projections are happening in industrial environments, and there are limited papers about these.
For example Klaus Birk\cite{Birken_Interactive} uses the graphical modelling extensions for hierarchical component models in systems architectures, as well as getting live simulator responses to changes in the code.
Tom Beadman\cite{Beadman_Journey}, as a part of his tool Pasta, generates an interactive visual state machine to, for example, decide whether to check for fraud. 

Florian Bocks tool, stiEF\cite{Bock_stief}, is a language that uses a live update of graphical driving scenarios in the IDE.
Whilst this is not a 2 way projection, in that a developer cannot interact with the visual scenarios.
However it is really a great feedback tool to let you know that the developer has correctly programmed their scenario. 