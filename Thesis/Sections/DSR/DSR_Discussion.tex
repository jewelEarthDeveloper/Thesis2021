\section{Discussion}
\label{section:dsr_discussion}

We now examine threats to the construct, the internal and external validity of our DSR, its reliability, and areas of improvement.


\subsection{Threats To Validity} 

\subsubsection{Conclusion Validity}
We felt that the projections we built improved our overview of large rule sets.
However, as we spent a lot of time with the rules as the projections are being built, we could have been influenced by the previous knowledge of the rules.
Thus, we needed to validate our conclusions with others, which chapter \ref{chapter:Survey} attempts to do.

\subsubsection{Construct Validity}
We were trying to observe whether projectional techniques could help with understanding Drools rules. 
As some of the outcomes could have also been achieved with other, non-projectional, language workbenches, these did not contribute to proving a link between better understanding and projectional editing.
As the tabular projections would not be achievable in parser based languages, then these could show the direct relationship between projections and understanding.

\subsubsection{Internal Validity}
If there was a better understanding, was this the result of the tabular projectional editing?
There are many advantages that come from the implementation in MPS, these include the context aware code completion menus.
Some of these side effects, rather than the core concept, could have impacted our understanding.

\subsubsection{External Validity}
Our Drools-Lite language is not a full implementation of the Drools language.
To take in the whole language would have taken more time.
Whether the examples we built would generalise to all the functionality of Drools is not known. 

\subsubsection{Reliability}
Reliability is how the data and the analysis is dependant on specific researchers.
With DSR studies, that build a working opensource prototype, then the reliability of the data is known.
The code can just be downloaded and run by any interested party.  

The reliability of our analysis, that the projections improved our understanding, is be open to effects of a range of cognitive biases.
To take out the effect of our biases this we surveyed others, as described in chapter \ref{chapter:Survey}.

\subsubsection{Repeatability Vs Reproducibility}
Implementing ideas in code can take

\subsubsection{Method Improvement}