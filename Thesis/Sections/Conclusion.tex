\chapter{Conclusion}\label{chapter:Conclusion}
In this master's project we dove into the current state of projectional editing and presented and had verified a number of projections for the purpose of improving the understanding of Business Rules.
We conducted a systematic literature review (SLR), creates prototypes and conducted a survey of the impact they may have.
Through the SLR we examined the current state of projectional editing, finding it dominated by a single product.
Through our prototypes we demonstrated some of the possibilities projectional editing can bring.
The results of our survey failed to confirm that these would bring the benefits we had hypothesised.

In this last section of this paper will return to the initial questions we were trying to answer from section \ref{section:Research_Questions}.
We will summarise the findings of our research and the contributions our work had made.
Finally we will discuss items that were out of scope for this project, but would give future research opportunities.


\section{Research Summary}
Our main research question, ``How can projectional editors and DSLs be combined to address feedback mechanisms for developers in the context of reasoning about rules in a rule-based business engine?'', motivated us to explore the field of projectional editing in this project.
This drove us to investigate if this field was worth investigating and, if so, how we can use existing tools to answer this.

\subsubsection{Research Question 1: ``What is the current state of language workbenches supporting projectional editing?''}
In this paper we presented an SLR which looked into the current state of projectional editing.
This SLR found that currently LWBs for projectional editing is a very narrow field, dominated by one product, JetBrains MPS.
Studies using products other than MPS spend time and effort discussing how to solve issues already solved in MPS.
Studies using MPS tend to be focused on products for industrial use or how to use MPS as a tool to teach Meta-Programming.
One area MPS is a little behind is the need for a web based development environment, both for the Language engineer, but especially for the developer.

A monoculture can be a risk.
Whilst there are many advantages to projectional editing, having only one successful product and supplier feels a little unhealthy.
On the other hand, the number of users of the tool is growing, as evidenced by the papers representing new projects in multiple industries, from hardware, through automotive industry, to Finance.

To conclude, the state of the projectional editing market is maturing but with a current product monoculture.

\subsubsection{Research Question 2: ``Which projections can help developers get appropriate feedback about rules?''}



\section{Summary of contributions}
\subsection{Primary contributions}
\subsection{Limitations of work}
\section{Further Work}

We have built our projections as an aid to the understanding of Drools rules.
This DSL extension includes many different ways to look at and interact with large code bases, as well as presenting options to deal with the complexity of individual rules.
This means that they must be [TODO: PROPERTIES THAT AIDS UNDERSTANDING].
Our questionnaires show that we have reached that aim.
Since developing our projections we have used them to model complex rules in our host organization.

Two factors lead to our success.
First was the flexibility and extensibility of the MPS tool which presented the ability to develop and extend DSLs very efficiently.
If we had tried this project without this tooling we would have [TODO: Finish our thoughts]
Second [TODO: Finish our thoughts]

In this paper we described our work with first translating the Drools DSL into a projectional language followed by our explorations of projections.
We discussed the advantages and disadvantages of the different projections we created and analysed experienced developers reactions to them.

Whilst we are convinced our projections [TODO: finish our thoughts]



\begin{itemize}
    \item more projections
    \begin{itemize}
        \item graphical
        \item data flow
        \item test output
    \end{itemize}
    \item FASTEN - formal specification/checkable
    \item an Experiment
\end{itemize}