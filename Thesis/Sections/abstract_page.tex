\begin{abstract}
\addchaptertocentry{\abstractname} 

\noindent
\textit{Context}: 
Declarative rules engine languages, such as Drools, can become difficult to reason about when there are many rules.

\noindent
\textit{Objective}: 
This project investigates whether projectional editing is a valid solution to this issue.
If so, how can different projections of the code can ease the comprehensibility of the code.

\noindent
\textit{Method}: 
We conducted a systematic literature review, to ascertain the relevancy of projectional editing.
We created an implementation of the Drools language using the MPS language workbench and made innovative projections of Drools ASTs.
We validated our projections with a survey.

\noindent
\textit{Results}:
Projectional editing is still niche, though it is making some industrial and educational headway in particular with JetBrains MPS.
Projections can be useful, though, our survey found that our projections did not outperform the understanding of textual representation at least amongst experience Drools users.

\noindent
\textit{Keywords}:
projectional editing; Rules Engines; MPS; Drools

\noindent
\textit{Paper type}:
Research paper
\end{abstract}
