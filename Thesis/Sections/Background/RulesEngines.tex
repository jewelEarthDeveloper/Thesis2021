\section{RulesEngines}

\subsection{What is a rules engine?}

In this section we will describe what a rules engine is and a little of its history.

The Aristotelian doctrine of essentialism declares that a thing has properties that are essential and properties that are accidental.
If one takes away accidental properties, then the thing remains the thing.
If one takes away essential properties, the thing is no longer the thing.
If the thing is a business application, then its essential properties are its business rules.


Simply put, business rules are the principles or regulations by which an organization carries out the tasks needed to achieve their goals.
When properly defined these rules can be encoded into statements that defines or constrains some aspect of the business organizational behaviour.
A rule consists of a condition and an action.
When the condition is satisfied then the action is performed.
More formally, business rules can be seen as the implication in the basic logical principle of Modus Ponens.

When described like this, one could me forgiven for thinking is this not just an if\-then logic that is frequently used in traditional programming.
One would not be wrong, however in traditional programming, representing all the combinatorial outcomes can become complex.
In the typical application architecture, rules are distributed in the source code or database.
Each additional rule leads to more fragility.

Documentation describing these rules may be found in the design documentation or user manuals.
However, as applications evolve documentation gets out of sync with codebase.
Once this desynchronization occurs, to know what the rules that govern the application, one has to navigate the codebase and decode the rules from their, often scattered, locations.

A rules engine is also known as a Business Rules Engine, a Business Rules Management System or a Production Rules System.
The goal of a rules engine  is the abstraction of business rules into encoded and packaged logic that defines the tasks of an organization with the accompanying tools that evaluate and execute these rules.
Simply put, they are where we evaluate our rules.
Rules engines match rules against facts and infer conclusions.
If we return to the Modus Ponens comparison:

\begin{tabular}{c@{\,}l@{}} 
    & $p$ \\
\arrayrulecolor{blue!60!green!70}    & $p \to q$ \\\cline{2-2}
$\therefore$         & $q$ \\
\end{tabular}

If the premise $p$ holds and the implication $p \to q$ holds then the conclusion $q$ holds.
In terms of a rule engine and business rules this could be seen as:
\begin{enumerate}
    \item the rules engine gathers the data for the premise: $p$
    \item it examines the business rules as the implications: $p \to q$
    \item it executes the conclusion: $q$
\end{enumerate}

Rules engines follow the recognize-act cycle.
First the match, i.e. are there any rules with a true condition, Next the do conflict resolution, to pick the most relevant of the matching rules, finally they act, which is to perform the actions described in the rule.
If no items are matched then the cycle is terminated, otherwise the first step is returned to.

Rules Engines are declarative, focussing on the what of the rules not the the how of the execution.
Date\cite{date2000not} describes rules engine as to ``specify business process declaratively, via business rules and get the system to compile those rules in to the necessary procedural (and executable) code.''
Fowler\cite{Fowler_rulesEngine} describes rules engine as follows: `` ... providing an alternative computational model.
Instead of the usual imperative model, which consists of commands in sequence with conditionals and loops, a rules engine is based on a Production Rule System.
This is a set of production rules, each of which has a condition and an action ...''.

Rule engines arose from the expert systems of the late 70s and early 80s.
Expert systems initially had three main techniques for knowledge representation: Rules, frames and logic\cite{jackson1986introduction}.
"The granddaddy" of the expert systems, MYCIN, relied heavily on rules based knowledge representation\cite{shortliffe1974mycin}, rather than long inference chains.
MYCIN was used to identify bacteria and recommend antibiotic prescriptions.
MYCIN and its progenitor, DENDRAL, spawned a whole family of Clinical Decision Support Systems that pushed the rules engine technology until the early 1980's.
Research into rules engines died out in the 1980s as it fell out of fashion.

Early in their existence, the rules engines hit a limiting factor because the matching algorithms they used suffered from the utility problem, i.e. the match cost increased linearly with the number of rules being examined/
This problem was solved by Charles Forgy's efficient pattern matching Rete algorithm\cite{forgy1989rete}, and its successors.
This algorithm works by modelling the rules as a network of nodes where each node type works as a filter.
A fact will be filtered through this network.
The pre-calculation of this network is what provides the performance characteristics.

The first popular rules engine was Office Production System from 1976 
In 1981 OPS5 added the Rete algorithm.
CLIPS in  ..
JESS
Drools 



In general, rules engines are forward chaining.
This means to test if 
[TODO: Explain forward chaining with logic symbols] 

[TODO: ADD MORE HISTORY HERE]

Moving forward to current times, there are a few rules engines currently in use.
Some of the more commonly used ones are shown in table \ref{table:RuleEngines}

\begin{table}
    \begin{center}
        \begin{tabular}{ |l c |l|l| } 
            \hline
            Product                      &                             & Developer    & licence type   \\
            \hline
            CLIPS                        & \cite{CLIPSProductPage}     & NASA         & open source    \\ 
            Drools                       & \cite{DroolsProductPage}    & JBoss/RedHat & open source    \\ 
            BizTalk Business Rule Engine & \cite{BiztalkProductPage}   & Microsoft    & proprietary    \\ 
            WebSphere ILOG JRules        & \cite{JRulesProductPage}    & IBM          & proprietary    \\ 
            OpenRules                    & \cite{OpenRulesProductPage} & OpenRules    & open source    \\ 
            \hline
        \end{tabular}
    \end{center}
    \caption{Rules Engine products.}
    \label{table:RuleEngines}
\end{table}

Some of the advantages of using a rules engine include:
\begin{itemize}
    \item The separation of knowledge from it's implementation logic
    \item Business logic can be externalized
    \item Rules can be human readable
\end{itemize}

% wip
Rules that represent policies are easily
communicated and understood.
Rules retain a higher level of independence than
conventional programming languages.
Rules separate knowledge from its implementation
logic.
Rules can be changed without changing source
code; thus, there is no need to recompile the
application's code.
Cost of production and maintenance decreases.
% end wip 


In summary a rules engine, is the executor of a rules based program, consisting of discreet declarative rules which model a part of the business domain.


\subsection{What is Drools?}\label{section:WhatIsDrools}

JBoss Rules, or as it is more commonly known, Drools, is the leading opensource rules engine written in Java.
In this paper when we use the name ``Drool'' we are referring to the ``Drools Expert'' which is the rule engine module of the Drools Suite.
Drools started in 2001, but rose to prominence with it's 2005 2.0 release.
It is an advanced inference engine using an enhanced version of the Rete algorithm, called Rete\-OO\cite{sottara2010configurable}, adapted to an object-oriented interface specifically for Java.
Designed to accept pluggable language implementations, it can also work with Python and .Net.
It is considered one of the most developed and supported rules platforms.

For rules to be executed there are 4 major components as demonstrated in figure \ref{fig:Drools_components}.
The production memory contains the rules.
This will not change during an analysis session.
The rules are the focus of this thesis and therefore we will delve into much more detail later on these.

In Forgy's\cite{forgy1989rete} overview of a rete algorithm, the following steps occur.
\begin{enumerate}
    \item Match : Evaluate the LHSs of the productions to determine which are satisfied given the current contents of working memory
    \item Conflict resolution : Select one production with a satisfied LHS; if no productions have satisfied LHSs, halt the interpreter
    \item Act : Perform the actions in the RHS of the selected production
    \item Re-evaluate : Go To 1
\end{enumerate}

Figure \ref{fig:Drools_inference_loop} show more detail of how these components interact within Drools to infer a conclusion.
First a fact or facts are asserted in the working memory.
The working memory contains the current state of the facts.
This triggers the inference engine.
The pattern matcher, using the aforementioned Rete\-OO algorithm will determine examine the working memory and a representation of the rules from the production memory to determine which rules are true.
Matching rules will be placed on the agenda.
It can be the case that many rules are concurrently true for the same fact assertion.
These rules are in conflict.
A conflict resolution strategy will decide which rule will fire in which order from the agenda.
The first rule on the agenda will fire.
If the rule modifies, retracts or asserts a fact, then the inference loop begins again.
If a rule specifies to halt or there are no matching rules left on the agenda, we have inferred our conclusion.


\begin{figure}[h]
    \centering
    \fbox{\includegraphics[width=0.55\textwidth]{Sections/images/components.png}}
    \caption{Drools components.}
    \label{fig:Drools_components}
\end{figure}

\begin{figure}[h]
    \centering
    \fbox{\includegraphics[width=0.55\textwidth]{Sections/images/InferenceLoop-1.png}}
    \caption{Drools Inference Loop.}
    \label{fig:Drools_inference_loop}
\end{figure}


The component we will be focussing on in this paper is the rules.
Rules are stored in a rules file, a text file, typically with a \.drl extension.
During execution the rules do not change and are stored in production memory.
For the sake of this paper we will skip past package, import, global, declare, function and query, which are also stored in the rule file.
We will examine the anatomy of a rule.

A rule is made of 3 parts: attributes; conditions; and consequences.
Attributes are an optional hints to the inference engine as to how the rule should be examined.
The conditional, when, or left hand side (LHS) of the rule statement is a block of conditions that have to in aggregate return true for the asserted fact in order to be considered to be placed on the agenda.
The actions, consequences, then, or Right hand side (RHS) of the rule statement contains actions to be executed, should the rule be filtered

The LHS is a predicate statement, made up of a number of patterns.
variables can be bound to facts that match these patterns for use later in the LHS or for updating the working memory on the RHS.
the patterns are used to evaluate against the working memory.
The pattern match against the existence of facts.
Patterns can also match against conditions of the properties of facts.
Connectives such as not, and, and or can be applied to the patterns.
The patterns apply to individual Facts rather than the group, thus can be seen as first order predicates.

There are some more advanced features in the LHS, but for this paper, these are the features we will be looking at.

Whilst the RHS can contain arbitrary code to be executed when a rule is fired, it's main purpose is to adjust the state of truth in the working memory.
One can insert, modify, and retract facts in the working memory.
modifying and retracting facts, must be done on fact variable references that have been created in the LHS.
One can explicitly terminate the inference loop, with a halt command.

\begin{figure}[h]
    \centering
    \fbox{\includegraphics[width=0.95\textwidth]{Sections/images/DroolsRule2.png}}
    \caption{Drools Rule Breakdown.}
    \label{fig:Drools_Rule_Breakdown}
\end{figure}


\subsubsection{An explanatory example}
An example of a drl file can be seen in listing \ref{listing:drl_file}.
This has been extracted from the Drools sample code.

\begin{lstlisting}[language={[drl]Drools}, caption=Example Drools file., captionpos=b, label=listing:drl_file]
    package org.drools.examples.honestpolitician
 
    import org.drools.examples.honestpolitician.Politician;
    import org.drools.examples.honest politician.Hope;
     
    rule ``We have an honest Politician''
        salience 10
        when
            exists( Politician( honest == true ) )
        then
            insertLogical( new Hope() );
    end
    
    rule ``Hope Lives''
        salience 10
        when
            exists( Hope() )
        then
            System.out.println(``Hurrah!!! Democracy Lives'');
    end
    
    rule ``Hope is Dead''
        when
            not( Hope() )
        then
            System.out.println( ``We are all Doomed!!! Democracy is Dead'' );
    end
    
    rule ``Corrupt the Honest''
        when
            $p : Politician( honest == true )   
            exists( Hope() )
        then
            System.out.println( ``I'm an evil corporation and I have corrupted `` + $p.getName() );
            modify( $p ) { 
                setHonest( false ) 
            }
    end
\end{lstlisting}

Listing \ref{listing:drl_file} gives the Drools engine instructions on what actions to take when something changes in the working memory.
What this toy example does is reacts to when an honest politician is added to the working memory, prints a message celebrating the existence of said politician, corrupts her, gloats in a message and then prints a message of despair.
The code in listing \ref{listing:drl_file} does the following: 
\begin{enumerate}[topsep=2pt,itemsep=2pt,partopsep=2pt, parsep=2pt]
    \item on line 1 the package statement identifies the rule file
    \item on lines 3 and 4 the import statements describes which facts can be used
    \item the ``We have an honest Politician'' rule on line 6 does the following:
    \begin{enumerate}[topsep=2pt,itemsep=2pt,partopsep=2pt, parsep=2pt]
        \item using salience on line 7 it sets that this rule is to be run before rules with a lower salience
        \item on line 10 it checks the working memory for Politician facts with the honest property equal to true
        \item on line 12, if found then Hope facts will be inserted into the working memory
    \end{enumerate}
    \item the ``Hope Lives'' rule on line 15 does the following:
    \begin{enumerate}[topsep=2pt,itemsep=2pt,partopsep=2pt, parsep=2pt]
        \item line 18 check if any Hope facts exist
        \item on line 20, if found, it prints a message
    \end{enumerate}
    \item the ``Hope is Dead'' rule on line 23 does the following:
    \begin{enumerate}[topsep=2pt,itemsep=2pt,partopsep=2pt, parsep=2pt]
        \item checks if no Hope facts exist on line 25
        \item if none are found, on line 27, it prints a message 
    \end{enumerate}
    \item the ``Corrupt the Honest'' rule on line 30 does the following:
    \begin{enumerate}[topsep=2pt,itemsep=2pt,partopsep=2pt, parsep=2pt]
        \item line 32 checks for any Politician facts with the honest property equal to true, and sets them to the variable \$p
        \item line 33 checks if any Hope facts exist
        \item if both hope and politicians are found on line 35 it prints a message including the \$p variables name
        \item on line 36 to 38 it modifies the fact in working memory represented by \$p to change it's honest property 
    \end{enumerate}
\end{enumerate}




