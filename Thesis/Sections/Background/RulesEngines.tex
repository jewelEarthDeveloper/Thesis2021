\section{RulesEngines}

\subsection{What is a rules engine?}

In this section we will describe what a rules engine is and a little of its history.

The Aristotelian doctrine of essentialism declares that a thing has properties that are essential and properties that are accidental.
If one takes away accidental properties, then the thing remains the thing.
If one takes away essential properties, the thing is no longer the thing.
If the thing is a business application, then its essential properties are its business rules.


Simply put, business rules are the rules by which an organization carries out the tasks needed to achieve their goals.
When properly defined these rules can be encoded.
In the typical application architecture, these rules are mostly distributed in the source code or database.
Documentation describing these rules may be found in the design documentation or user manuals.
However, as applications evolve documentation gets out of sync with codebase.
Once this desynchronization occurs, to know what the rules that govern the application, one has to navigate the codebase and decode the rules from their, often scattered, locations.


Rule engines arose from the expert systems of the late 70s and early 80s. 
Expert systems initially had three main techniques for knowledge representation: Rules, frames and logic\cite{jackson1986introduction}.  
"The granddaddy" of the expert systems, MYCIN, relied heavily on rules based knowledge representation\cite{shortliffe1974mycin}, rather than long inference chains.
MYCIN was used to identify bacteria and recommend antibiotic prescriptions.
MYCIN and its progenitor, DENDRAL, spawned a whole family of Clinical Decision Support Systems that pushed the rules engine technology until the early 1980's.
Research into rules engines died out in the 1980s as it fell out of fashion.


[TODO: ADD MORE HISTORY HERE]


The goal of a rules engine is the abstraction of business rules into encoded and packaged logic that defines the tasks of an organization.
Date\cite{date2000not} defines the concept of the rules engine as to ``specify business process declaratively, via business rules and get the system to compile those rules in to the necessary procedural (and executable) code.''
Fowler\cite{Fowler_rulesEngine} describes rules engine as follows: `` ... providing an alternative computational model.
Instead of the usual imperative model, which consists of commands in sequence with conditionals and loops, a rules engine is based on a Production Rule System.
This is a set of production rules, each of which has a condition and an action ...''.

[TODO: ADD MORE ADVANTAGES HERE]


In summary a rules engine, is the executor of a rules based program, consisting of discreet declarative rules which model a part of the business domain.

\subsection{What is Drools?}




