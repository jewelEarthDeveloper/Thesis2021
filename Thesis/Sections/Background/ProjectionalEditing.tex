\section{Projectional Editing}\label{chapter:projectional_editing}
\subsection{What is projectional editing?}

\subsection{what are Language Workbenches?}

\subsection{What is MPS?}

% Notes:

% * Projectional editing sometimes (sometimes called structured) editing can offer a wider range of notations.

% Symbols - notations that combine text with non-textual symbols. Math, for example, has been using symbols for centuries and there are good reasons to support these symbols in languages used in math-heavy computations.

% State machines - some domains use state machines to model systems that react to external events and change their internal state following some pre-defined set of rules. To express the transitions in individual states after arrival of certain events a table can be used with good success.

% Diagrams - numerous problems are best solved in graphics - electrical circuits, organizational charts, workflows, etc.

% Forms - rich text editors are also frequently utilized for encoding rules of various sorts.

%  notations can be optimized towards easy readability or towards writability

%  The ability to have multiple notations for a single language, which can be switched when editing code, the trade off between learnability and effectivity as well as between readability and writability does not need to be made in MPS languages.

%  3 benefits of projectional editors
%  1) richness of the notations you can use in your languages
%  2) change notations on the fly, so you can choose the notation that best suits the problem at hand
%  3) combine languages easily
 