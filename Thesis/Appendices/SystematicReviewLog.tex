\chapter{Systematic Literature Review Log} \label{Appendix:SLRLog}

\section*{Search Description} 
The papers we found with our automatic search can be found in table \ref{table:Systematic_Review_Log_1}.
There were 100 unique papers found with the search strings across all of the venues when we used the search string and date restrictions.
First we excluded those that were not in English or were unavailable to us.
This reduced the count to 94.
Next we downloaded each of the papers.

Next we skimmed all of these papers to remove any that were obviously unrelated to projectional editing.
This reduced the count to 69.

two of the papers were referring to the same study, which reduced the count to 67.

We then excluded all grey literature, i.e. masters projects, proposals and PhD theses, and also books.
This brought us down to 51 papers.

At this point we began our quality assessment.

\section*{Table description}
Table \ref{table:Systematic_Review_Log_1} shows the log of the Systematic literature review.

The First column ``Paper Title'' is the name of the paper as given by the search engine.

The second column ``Lib'', indicates the library or search engine through which it was found.
The Libraries are Identified by the Keys in table \ref{table:SearchEngineKey}.


\begin{table}[h]
    \begin{center}
        \begin{tabular}{ | l | l | l | l |} 
            \hline
            Key & Search engine/library    & Key & Search engine/library     \\
            \hline
            \hline
            1  & Google Scholar            & 2  & IEEExplores                \\
            3  & ACM                       & 4  & BASE                       \\
            5  & CORE                      & 6  & Web of Science             \\
            7  & Microsoft Academic        & 8  & SCOPUS                     \\
            9  & Semantic Scholar          & 10 & SpringerLink               \\
            11 & Wiley Online              & 12 & Science.gov                \\
            \hline
        \end{tabular}
    \end{center}
    \caption{Search Engine/Library Key}
    \label{table:SearchEngineKey}
\end{table}

The third and fourth columns show inclusion and exclusion reasons.
As inclusions only rely on one question, ``does this paper discuss projectional editing'', the affirmative is indicated by a tick.
The exclusion column includes the reason for  the exclusion.

The final column ``ref'', gives a link to the citations in the separate bibliography for the systematic literature review, found at the end of the Appendices.

\begin{landscape}
    \begin{longtable}{ | p{15cm} | *{5}{c|} }
    \hline
    Paper Title                                                                                                                                                   & lib       & in     &  exclusion  & F\# & B\# \\ \hline 
    \hline
    \endhead  % header material
    \hline\endfoot  % footer material
        ``Filmar, assistir e problematizar'' – contribuições à aprendizagem de cálculos                                                                           & 9         &        & not English &  X  & X   \\ \hline 
        20. Internationales Stuttgarter Symposium                                                                                                                 & 10        &        & book        &  X  & X   \\ \hline 
        A Domain-Specific Language for Payroll Calculations: a Case Study at DATEV                                                                                & 1         & \cmark &             &  0  & 0   \\ \hline 
        A Domain-Specific Language for Payroll Calculations: An Experience Report from DATEV                                                                      & 1,10      & \cmark & Duplicate   &  X  & X   \\ \hline 
        A Framework for Modernizing Domain-Specific Languages                                                                                                     & 1         & \cmark & grey        &  X  & X   \\ \hline 
        A Framework for Projectional Multi-variant Model Editors                                                                                                  & 1,8       & \cmark &             &  0  & 0   \\ \hline 
        A Generic Projectional Editor for EMF Models                                                                                                              & 1,7,8,9   & \cmark &             &  2  & 0   \\ \hline 
        A language-driven Development framework for simulation components to generate simulated environments                                                      & 1         & \cmark & grey        &  X  & X   \\ \hline 
        A Model-Driven Approach Towards Automatic Migration to Microservices                                                                                      & 10        & \cmark &             &  5  & 0   \\ \hline 
        A survey of Model Driven Engineering in robotics                                                                                                          & 1         & \cmark &             &  2  & 2   \\ \hline 
        A Survey on the Design Space of End-User Oriented Languages for Specifying Robotic Missions                                                               & 1,10      & \cmark &             &  1  & 5   \\ \hline 
        A survey on the formalisation of system requirements and their validation                                                                                 & 1         & \cmark &             &  0  & 0   \\ \hline 
        A text-based syntax completion method using LR parsing                                                                                                    & 1         &        &             &  X  & X   \\ \hline 
        Activities and costs of re-engineering cloned variants into an integrated platform                                                                        & 1         &        &             &  X  & X   \\ \hline 
        AdaptiveVLE: An Integrated Framework for Personalized Online Education Using MPS JetBrains Domain-Specific Modeling Environment                           & 1,2       & \cmark &             &  1  & 1   \\ \hline 
        Adding Interactive Visual Syntax to Textual Code                                                                                                          & 3         & \cmark &             &  3  & 0   \\ \hline 
        An approach to generate text-based IDEs for syntax completion based on syntax specification                                                               & 1         & \cmark &             &  1  & 0   \\ \hline 
        An MPS implementation for SimpliC                                                                                                                         & 1         & \cmark & grey        &  x  & x   \\ \hline 
        Blended graphical and textual modelling for UML profiles: A proof-of-concept implementation and experiment                                                & 1         & \cmark &             &  0  & 0   \\ \hline 
        Block-based syntax from context-free grammars                                                                                                             & 1,3,5     & \cmark &             &  0  & 2   \\ \hline 
        Bridging the worlds of textual and projectional language workbenches                                                                                      & 1         & \cmark & grey        &  X  & X   \\ \hline 
        Classification Algorithms Framework (CAF) to Enable Intelligent Systems Using JetBrains MPS Domain-Specific Languages Environment                         & 2,5       & \cmark &             &  4  & 0   \\ \hline 
        Code and Structure Editing for Teaching: A Case Study in using Bibliometrics to Guide Computer Science Research                                           & 1,9       & \cmark &             &  2  & 0   \\ \hline 
        ComPOS - a Domain-Specific Language for Composing Internet-of-Things Systems                                                                              & 1,4       & \cmark & grey        &  X  & X   \\ \hline 
        Concepts of variation control systems                                                                                                                     & 1         & \cmark &             &  9  & 5   \\ \hline 
        Concise, Type-Safe, and Efficient Structural Diffing                                                                                                      & 3         &        &             &  X  & X   \\ \hline 
        Constructing optimized constraint-preserving application conditions for model transformation rules                                                        & 1         &        &             &  X  & X   \\ \hline 
        Design \& Evaluation of an Accessible High-Level Language for Advanced Cryptography                                                                       & 1         & \cmark & grey        &  X  & X   \\ \hline 
        Domain-specific languages for modeling and simulation                                                                                                     & 1         &        &             &  X  & X   \\ \hline 
        Domain-Specific Languages in Practice                                                                                                                     & 10        & \cmark & book        &  X  & X   \\ \hline 
        DSL Based Approach for Building Model-Driven Questionnaires                                                                                               & 1,10      & \cmark &             &  0  & 1   \\ \hline 
        DSS-Based Ontology Alignment in Solid Reference System Configuration                                                                                      & 10        &        & unavailable &  X  & X   \\ \hline 
        Editing Software as Strategy Value                                                                                                                        & 1         &        &             &  X  & X   \\ \hline 
        Efficient editing in a tree-oriented projectional editor                                                                                                  & 1,3,7,8,9 & \cmark &             &  1  & 0   \\ \hline 
        Efficient generation of graphical model views via lazy model-to-text transformation                                                                       & 1,4       & \cmark &             &  1  & 0   \\ \hline 
        Efficient usage of abstract scenarios for the development of highly-automated driving functions                                                           & 1,10      & \cmark & unavailable &  X  & X   \\ \hline 
        Enabling language engineering for the masses                                                                                                              & 1,3       & \cmark &             &  2  & 1   \\ \hline 
        Engineering Gameful Applications with MPS                                                                                                                 & 1,10      & \cmark &             &  0  & 2   \\ \hline 
        Enhancing development and consistency of UML models and model executions with USE studio                                                                  & 1,3,7,8,9 & \cmark &             &  0  & 2   \\ \hline 
        Enterprise Information Systems                                                                                                                            & 10        &        & book        &  X  & X   \\ \hline 
        Example-driven software language engineering                                                                                                              & 1,3       & \cmark &             &  1  & 2   \\ \hline 
        Exploring Visual Primitives for Authoring Source Code                                                                                                     & 1         &        & grey        &  X  & X   \\ \hline 
        FASTEN: An Extensible Platform to Experiment with Rigorous Modeling of Safety-Critical Systems                                                            & 1,10      & \cmark &             &  0  & 5   \\ \hline 
        FeatureCoPP: unfolding preprocessor variability                                                                                                           & 1,3       &        &             &  X  & X   \\ \hline 
        FeatureVista: Interactive Feature Visualization                                                                                                           & 1         &        &             &  X  & X   \\ \hline 
        Filling Typed Holes with Live GUIs                                                                                                                        & 3         & \cmark &             &  0  & 5   \\ \hline 
        First-class concepts: reifying architectural knowledge beyond the dominant decomposition                                                                  & 1,3       & \cmark &             &  0  & 1   \\ \hline 
        FORMREQ 2020                                                                                                                                              & 1,2,8     &        & book        &  X  & X   \\ \hline 
        Gentleman: a light-weight web-based projectional editor generator                                                                                         & 1,3,4,7,8,9 & \cmark &           &  0  & 0   \\ \hline 
        GPP, the Generic Preprocessor                                                                                                                             & 1         &        &             &  X  & X   \\ \hline 
        Improving the usability of the domain-specific language editors using artificial intelligence                                                             & 1         & \cmark & grey        &  X  & X   \\ \hline 
        Incremental Flow Analysis through Computational Dependency Reification                                                                                    & 1,2       & \cmark &             &  0  & 2   \\ \hline    
        Incrementalizing Static Analyses in Datalog                                                                                                               & 1         & \cmark & grey        &  X  & X   \\ \hline 
        Integrating the Common Variability Language with Multilanguage Annotations for Web Engineering                                                            & 1         &        &             &  X  & X   \\ \hline 
        Integrating UML and ALF: An Approach to Overcome the Code Generation Dilemma in Model-Driven Software Engineering                                         & 10        & \cmark &             &  0  & 0   \\ \hline 
        Javardise: a structured code editor for programming pedagogy in Java                                                                                      & 1         & \cmark &             &  0  & 0   \\ \hline 
        JetBrains MPS as Core DSL Technology for Developing Professional Digital Printers                                                                         & 1,10      & \cmark &             &  0  & 0   \\ \hline 
        JetBrains MPS: Why Modern Language Workbenches Matter                                                                                                     & 1,7,10    & \cmark &             &  0  & 1   \\ \hline 
        Learning Data Analysis with MetaR                                                                                                                         & 1,10      & \cmark &             &  0  & 0   \\ \hline 
        Lipschitz-like property relative to a set and the generalized Mordukhovich criterion                                                                      & 6         &        &             &  X  & X   \\ \hline 
        Macros for Domain-Specific Languages                                                                                                                      & 3         &        &             &  X  & X   \\ \hline 
        Mechanizing metatheory interactively                                                                                                                      & 1         & \cmark & grey        &  X  & X   \\ \hline 
        Migrating Insurance Calculation Rule Descriptions from Word to MPS                                                                                        & 1,10      & \cmark &             &  0  & 0   \\ \hline 
        Model Driven Software Engineering Meta-Workbenches: An XTools Approach                                                                                    & 1,5       & \cmark &             &  0  & 0   \\ \hline 
        Model-based safety assessment with SysML and component fault trees: application and lessons learned                                                       & 1,10      & \cmark &             &  8  & 0   \\ \hline 
        Model-Driven Development for Spring Boot Microservices                                                                                                    & 1,5       & \cmark & grey        &  X  & X   \\ \hline 
        n Challenges for Software Language Engineering                                                                                                            & 1         &        &             &  X  & X   \\ \hline 
        On preserving variability consistency in multiple models                                                                                                  & 1,3       &        &             &  X  & X   \\ \hline 
        On the Need for a Formally Complete and Standardized Language Mapping between C++ and UML                                                                 & 1         &        &             &  X  & X   \\ \hline 
        On the Understandability of Language Constructs to Structure the State and Behavior in Abstract State Machine Specifications: A Controlled Experiment     & 1         &        &             &  X  & X   \\ \hline 
        On the use of product-line variants as experimental subjects for clone-and-own research: a case study                                                     & 1         &        &             &  X  & X   \\ \hline 
        PAMOJA: A component framework for grammar-aware engineering                                                                                               & 1         & \cmark &             &  0  & 1   \\ \hline 
        Programming Robots for Activities of Everyday Life                                                                                                        & 1         & \cmark & grey        &  X  & X   \\ \hline 
        Programming tools for intelligent systems                                                                                                                 & 1         & \cmark & grey        &  X  & X   \\ \hline 
        Projecting Textual Languages                                                                                                                              & 1,10      & \cmark &             &  0  & 1   \\ \hline 
        Rule-based and user feedback-driven decision support system for transforming automatically-generated alignments into information-integration alignments   & 5         &        & unavailable &  X  & X   \\ \hline 
        Semi-Automatische Deduktion von Feature-Lokalisierung während der Softwareentwicklung: Masterarbeit                                                       & 5         &        & not English &  X  & X   \\ \hline 
        Should Variation Be Encoded Explicitly in Databases?                                                                                                      & 1         &        &             &  X  & X   \\ \hline 
        SLang: A Domain-specific Language for Survey Questionnaires                                                                                               & 1         & \cmark &             &  0  & 0   \\ \hline 
        SpecEdit: Projectional Editing for TLA+ Specifications                                                                                                    & 1,4,7     & \cmark &             &  0  & 0   \\ \hline 
        Specifying Software Languages: Grammars, Projectional Editors, and Unconventional Approaches                                                              & 1,2,4,5,7,8,9& \cmark &             &  0  & 6 \\ \hline 
        Teaching Language Engineering Using MPS                                                                                                                   & 10        & \cmark &             &  0  & 0   \\ \hline 
        Teaching MPS: Experiences from Industry and Academia                                                                                                      & 1,10      & \cmark &             &  0  & 1   \\ \hline 
        Teasy framework: uma solução para testes automatizados em aplicações web                                                                                  & 1         & \cmark & not English &  X  & X   \\ \hline 
        The Art of Bootstrapping                                                                                                                                  & 10        & \cmark &             &  3  & 0   \\ \hline 
        The state of adoption and the challenges of systematic variability management in industry                                                                 & 1         &        &             &  X  & X   \\ \hline 
        Toward a domain-specific language for scientific workflow-based applications on multicloud system                                                         & 1,11      &        &             &  X  & X   \\ \hline 
        Towards a Universal Variability Language                                                                                                                  & 1         & \cmark & grey        &  X  & X   \\ \hline 
        Towards Multi-editor Support for Domain-Specific Languages Utilizing the Language Server Protocol                                                         & 10        & \cmark &             &  5  & 0   \\ \hline 
        Towards Ontology-based Domain Specific Language for Internet of Things                                                                                    & 1,3       & \cmark &             &  0  & 0   \\ \hline 
        Towards projectional editing for model-based SPLs                                                                                                         & 3,4,7,8,9 & \cmark &             &  3  & 0   \\ \hline 
        Tychonis: A model-based approach to define and search for geometric events in space                                                                       & 1         &        &             &  X  & X   \\ \hline 
        Type-Directed Program Transformations for the Working Functional Programmer                                                                               & 1         & \cmark &             &  0  & 0   \\ \hline 
        Understanding Variability-Aware Analysis in Low-Maturity Variant-Rich Systems                                                                             & 1         &        &             &  X  & X   \\ \hline 
        Untangling Mechanized Proofs                                                                                                                              & 3         &        &             &  X  & X   \\ \hline 
        Variability representations in class models: An empirical assessment                                                                                      & 1,3       &        &             &  X  & X   \\ \hline 
        Visual design for a tree-oriented projectional editor                                                                                                     & 1,3,4,7,8,9& \cmark & Duplicate  &  X  & X   \\ \hline 
        What do practitioners expect from the meta-modeling tools? A survey                                                                                       & 1,7,8,9   & \cmark &             &  0  & 3   \\ \hline 
        Cyrillic named paper 1                                                                                                                                    & 1         &        & not English &  X  & X   \\ \hline 
        Cyrillic named paper 2                                                                                                                                    & 1         &        & not English &  X  & X   \\ \hline 
        \caption{Systematic review log - search results}
        \label{table:Systematic_Review_Log_1}
    \end{longtable}
\end{landscape}


